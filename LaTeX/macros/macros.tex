%%
%% LENGTHS
%%
\newlength{\exampleboxwidth} \setlength{\exampleboxwidth}{0.95 \textwidth}

%%
%% COUNTERS
%%
\newcounter{subexample}

%%
%% SYNTAX BOX
%%
\newcommand{\Msynbox}[1] {\vspace{0.5cm} SYNTAX : \fbox
  {\begin{minipage}[t]{13.5cm} #1 \end{minipage} }}

%%
%% BOX
%%
\newcommand{\Mbox}[1]{%
  \begin{boxedminipage}[t]{\textwidth}%
    #1%
  \end{boxedminipage}}

%%
%% LISA CODE EXAMPLE EVIRONMENT
%%
\newenvironment{lisaexample}[1][htbp]%
{\begin{example}[#1]
 \begin{center}%
 \vspace{1.3ex}
 \begin{boxedminipage}[t]{\exampleboxwidth}%
 \vspace{1.5ex}%
 \begin{center}
 \begin{minipage}[t]{0.94 \exampleboxwidth}
 \scriptsize
 %\footnotesize
}
{\end{minipage}%
 \end{center}%
 \vspace{-0.8ex}%
 \end{boxedminipage}%
 \end{center}
 \vspace{-1.2ex}
 \end{example}}

\newfloat{example}{tbphH}{loe}[chapter]
 \floatstyle{plain}
 \floatname{example}{Example}
 \floatplacement{example}{h}

\newcommand{\listofexamples}{\listof{example}{List of Examples}}

%%
%% RULE ENVIRONMENT
%%
\addtolength{\theorempreskipamount}{1.5ex}
\addtolength{\theorempostskipamount}{1.5ex}
\theorembodyfont{\slshape}
\theoremheaderfont{\bfseries\slshape}
\newtheorem{codingrule}{Rule}

%%
%% LIST
%%
\newcommand{\entrylabel}[1]{\mbox{\texttt{#1}}\hfil}
\newenvironment{cmdlist}%
{\begin{list}{}%
    {\renewcommand{\makelabel}{\entrylabel}%
     \setlength{\labelwidth}{40pt}
     \setlength{\leftmargin}{\labelwidth}
     \addtolength{\leftmargin}{\labelsep}%
    }%
}%
{\end{list}}

%%
%% RESERVED NAMES
%%

\newcommand{\keyword}{\sc}

\newcommand{\lisa}{{\keyword Lisa}}
\newcommand{\resource}{{\keyword Resource}}
\newcommand{\pipeline}{{\keyword Pipeline}}
\newcommand{\alias}{{\keyword Alias}}
\newcommand{\operation}{{\keyword Operation}}
\newcommand{\isin}{{\keyword In}}
\newcommand{\behavior}{{\keyword Behavior}}
\newcommand{\uses}{{\keyword Uses}}
\newcommand{\usein}{{\keyword In}}
\newcommand{\useout}{{\keyword Out}}
\newcommand{\useinout}{{\keyword Inout}}
\newcommand{\require}{{\keyword Require}}
\newcommand{\expression}{{\keyword Expression}}
\newcommand{\activation}{{\keyword Activation}}
\newcommand{\instruction}{{\keyword Instruction}}
\newcommand{\coding}{{\keyword Coding}}
\newcommand{\at}{{\keyword At}}
\newcommand{\syntax}{{\keyword Syntax}}
\newcommand{\semantics}{{\keyword Semantics}}
\newcommand{\declare}{{\keyword Declare}}
\newcommand{\group}{{\keyword Group}}
\newcommand{\reference}{{\keyword Reference}}
\newcommand{\instance}{{\keyword Instance}}
\newcommand{\llabel}{{\keyword Label}}
\newcommand{\main}{{\keyword main}}
\newcommand{\switch}{{\keyword Switch}}
\newcommand{\ifthenelse}{{\keyword If-Then-Else}}
\newcommand{\switchcase}{{\keyword Switch-Case}}      
\newcommand{\bytes}{{\keyword Bytes}}
\newcommand{\enum}{{\keyword Enum}}
\newcommand{\register}{{\keyword Register}}
\newcommand{\programcounter}{{\keyword Program\_Counter}}

%%\newcounter{lec}[chapter]
%%\newlength{\examplewidth} \setlength{\examplewidth}{0.9 \textwidth}
%%
%% FIGURE REFERENCE
%%
%%\newcommand{\Mfigref}[1]{(see figure \ref{fig:#1})}
%%
%% FIGURE
%%
%%\newcommand{\Mfig}[2]{
%%  \begin{figure}[htbp] 
%%    \centerline{\epsfig{file=eps/#1.eps,width= \textwidth,angle=0}}
%%    \caption{#2}\label{fig:#1}
%%  \end{figure} }
%%
%% SCALED FIGURE
%%
%%\newcommand{\Mfigscale}[3]{
%%  \begin{figure}[ht] 
%%    \centerline{\epsfig{file=eps/#1.eps,width=#3 \textwidth,angle=0}}
%%    \caption{#2}\label{fig:#1}
%%  \end{figure} }
%%
%% SCALED FIGURE FIXED
%%
%%\newcommand{\Mfigscalefix}[4]{
%%  \begin{figure}[#4] 
%%    \centerline{\epsfig{file=eps/#1.eps,width=#3 \textwidth,angle=0}}
%%    \caption{#2}\label{fig:#1}
%%  \end{figure} }
%%
%% FIGURE LANDSCAPE
%%
%%\newcommand{\Mfigland}[2]{
%%  \begin{figure}[htbp] 
%%    \centerline{\epsfig{file=eps/#1.eps,angle=90,height= \textheight}}
%%    \caption{#2}\label{fig:#1}
%%  \end{figure} }
%%
%% LISA CODE EXAMPLE
%%
%\newcommand{\lisacode}[2]{
%  \begin{center} 
%    \vbox{
%      \vspace{1.3ex}                        % vertikaler Abstand zum Text aussen
%      \fbox{
%        \parbox{1.2ex}{~}\hfill             % horizontale Einr"uckung
%        \begin{minipage}{\examplewidth}
%          \vspace{1.5ex}                    % vertikaler Abstand innen
%          {\footnotesize {\input{#1}}}
%          %%{\scriptsize {\input{examples/#1}}}
%          \vspace{-0.8ex}                   % vertikaler Abstand innen
%          \refstepcounter{lec}
%          \label{lex:#1}
%        \end{minipage}
%        }
%      \vspace{4ex}\\                      % vertikaler Abstand zum Text aussen
%      \centerline{Example \thechapter.\arabic{lec}: #2}
%      \index{[\thechapter.\arabic{lec}] #2} }
%  \end{center}}
%%
%% LISA CODE EXAMPLE REFERENCE
%%
%%\newcommand{\Mexref}[1]{\thechapter.\ref{lex:#1}}



%%% Local Variables: 
%%% mode: latex
%%% TeX-master: "main"
%%% End: 
