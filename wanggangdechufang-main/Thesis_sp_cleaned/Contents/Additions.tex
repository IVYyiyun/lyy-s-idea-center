% --------------------------------------------------------------------------
% 		Einheiten / Units
% --------------------------------------------------------------------------
\DeclareSIUnit[]\kmh{\kilo\meter\per\hour}
\DeclareSIUnit[]\ms{\meter\per\second}
\DeclareSIUnit[]\mss{\meter\per\square\second}
\DeclareSIUnit[]\qm{\square\meter}
\DeclareSIUnit\bar{bar}

% --------------------------------------------------------------------------
% 	Eigene Befehle / Own commands
% --------------------------------------------------------------------------

% Differenzialoperator / Differential operator 
\newcommand*{\diff}{\mathop{}\!\mathrm{d}}

% --------------------------------------------------------------------------
% 		vorgegebene Trennung von Woertern / predefined seperation of words 
% --------------------------------------------------------------------------
\hyphenation{
    ex-amp-le
    Fahrt-ab-brü-che
    Web-app
    Da-rauf-hin
% Ist"=Ge"-schwin"-dig"-keit % "= ist erst nach \begin{document} verfübar
}


% --------------------------------------------------------------------------
% 		Pakete / Packages 
% --------------------------------------------------------------------------

% Die folgenden Pakete wurden bereits in igm.sty / the following packages were already included in igm.sty

% {scrhack}	 				% Zusammenspiel von einigen Paketen mit KOMA-Script / Interaction of several packages with the KOMA-Script
% [utf8]{inputenc}    		% Input-Encodung / Input-encoding
% {babel}          			% Rechtschreibunterstuetzung / Spell aid
% {csquotes} 				% Deutsche Anfuehrungszeichen / German quotation marks
% [T1]{fontenc}         	% T1-kodierte Schriften, korrekte Trennmuster fuer Worte mit Umlauten / T1-encoded fonts, correct seperation of words with umlauts
% {lmodern}					% Schoenere Schrift / Nicer font
% {textcomp}				% Sonderzeichen im Text (z.B. €) / Special characters in the text (e.g. €)
% {textgreek}				% Griechische Symbole im Text / Greek symbols in the text
% {setspace}        		% Zeilenabstand / Line spacing
% {scrpage2}				% Kopf- und Fusszeilen / Header and footer
% {caption}       			% mehrzeilige Captions ausrichten / adjust multiline captions
% {booktabs}		 		% Schoene horizontale Linien / horizontal lines
% {multirow}		 		% Spalten und Zeilen weiter unterteilen / Divide lines and columns further
% {rccol}			 		% Ausrichtung von Spalten am Dezimalzeichen / Align columns according to the decimal point
% {graphicx, psfrag} 		% Zum Einbinden von Grafiken / Incorporation of graphics
% {subcaption}          	% Unterabbildungen / Sub-illustrations
% {amsmath}  				% Fuer erweiterte mathematische Konstrukte / for complex mathematic constructions
% {mathtools}				% Fuer Mathematikformeln: Indizes oben links / Mathematical formula: Indeces top left
% {mathrsfs,amssymb}		% Fuer Mathematikformeln: Symbole / Mathematical formula: Symbols
% {amsfonts}				% Fuer Mathematikformeln: Schriften / Mathematical formula: Fonts
% {bm}						% Fettschrift fuer Matrizen und Vektoren / Bold lettering for matrices and vectors
% {arydshln}				% Linien fuer Matrizen und Vektoren / lines for matrices and vectors
% {biblatex}				% Quellenangaben / Citations
% {acronym}					% For the list of abbreviations
% {siunitx}					% Für schöne Einheiten 
% {pdflscape}       		% Seiten im Querformat im PDF richtig anzeigen / Display pages properly in landscape mode
% {hyperref}				% PDF mit Hyperlinks
% {geometry}				% margins

% weitere Pakete können eingebunden werden / additional packages can be included

%\usepackage{pgfplots}						% Zum Erstellen von Vektorgrafiken / Creation of vector graphics
%\pgfplotsset{compat=1.13}						
%\usetikzlibrary{external,positioning,calc,decorations.markings,arrows,shapes,patterns}
%\tikzexternalize
%\newcommand{\includetikz}[1]{
%	\tikzsetnextfilename{Abbildungen/AbbildungenKompiliert/#1}%
%	\input{Abbildungen/AbbildungenTIKZ/#1.tikz}}%


% ganzes PDF einbinden können:
\usepackage{pdfpages}

% SPJ:
% Einrückungen im Inhaltsverzeichnis anpassen:
%\makeatletter
%\renewcommand*\l@section{\@dottedtocline{1}{1.5em}{2.3em}}
%\renewcommand*\l@subsection{\@dottedtocline{2}{3.8em}{3.2em}}
%\renewcommand*\l@subsubsection{\@dottedtocline{3}{7.0em}{4.1em}}
%\renewcommand*\l@paragraph{\@dottedtocline{4}{10em}{5em}}
%\renewcommand*\l@subparagraph{\@dottedtocline{5}{12em}{6em}}
%\makeatother


% SPJ:
% serifenlose Schrift verwenden:
\usepackage[scaled=0.92]{helvet} % Load Helvetica as default sans-serif font (from PS-Font collection)
\renewcommand{\familydefault}{\sfdefault}
\setkomafont{chapter}{\bfseries\large}
\setkomafont{section}{\bfseries}
\setkomafont{subsection}{\bfseries}
\setkomafont{subsubsection}{\bfseries}
\renewcommand*{\sectfont}{\bfseries}
\sisetup{detect-all} % Sans für Units nutzen

\sisetup{locale = DE} % Komma bei SI-Units

% 4. Ebene mit Nummern + im Inhaltsverzeichnis:
%\setcounter{secnumdepth}{4}
%\setcounter{tocdepth}{4}

% Tabellen über Seitenbreite
\usepackage{tabularx}
\usepackage{array}

% SVG-Dateien einbinden können (Vektordateien)
\usepackage{svg}

% Texte als Monospace formatieren, ohne Sonderzeichen als Latex zu interpretieren (z.B. _) Kurzform: verb+text+
% \usepackage{verbatim}

% Zeilenumbrüche mit Silbentrennung in \texttt Umgebungen:
\usepackage[htt]{hyphenat}

% ifas Zitations-Stil imitieren: /xxx/ statt [xxx]
\renewcommand{\mkbibbrackets}[1]{/#1/}

% use of \begin{description}[style=nextline]
\usepackage[shortlabels]{enumitem}

% linebreak with \\ within tabular
%  \usepackage{pbox}

% Blindtexte einfügen können
%\usepackage{blindtext}
%\usepackage{lipsum}

% Kopfzeile auch bei Kapitelanfängen:
\renewcommand*{\chapterpagestyle}{scrheadings}

\usepackage{layouts} % show lengths. eg:
% \the\textwidth
% textwidth in cm: \printinunitsof{cm}\prntlen{\textwidth}\\
% textwidth in inches: \printinunitsof{in}\prntlen{\textwidth}


\DeclareGraphicsExtensions{.pdf,.jpg,.png} % Rangfolge von Dateitypen
\graphicspath{{Contents/Resources}} % Suchpfad für Dateien

% Hidden table-column:
\newcolumntype{H}{>{\lrbox0}c<{\endlrbox}@{}}

% PDF-Layout: (default zweiseitig anzeigen)
% \pdfcatalog{/PageLayout /TwoPageRight}

% A newly defined length for negative vspace in the description-env:
\newlength{\negspacelength}
\setlength{\negspacelength}{-\baselineskip-\parskip}

\BeforeStartingTOC[toc]{\setstretch{1.19}} % Inhaltsverzeichnis auf eine Seite quetschen, eine Zeile mehr passt mit 1.08


\RedeclareSectionCommand[%
    beforeskip=0pt,
    afterskip=6pt
]{chapter}
\RedeclareSectionCommand[
    beforeskip=6pt,
    afterskip=6pt
]{section}
\RedeclareSectionCommand[
    beforeskip=6pt,
    afterskip=6pt
]{subsection}
\RedeclareSectionCommand[
    beforeskip=6pt,
    afterskip=6pt
]{subsubsection}
